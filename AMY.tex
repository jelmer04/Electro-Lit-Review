\section{Data Logging}

\subsection{Memory}
Different techniques can be used for storing digital data, which can be split into two categories. Volatile memory requires constant power to maintain saved data, but if the power is interrupted, the data is lost. Volatile memory is very fast to save to and recover from, but due to the need for constant power, is not used when data is to be stored for long periods of time. Non-volatile memory is not lost when power to the memory device is removed, and is therefor a much more reliable method of storing important data. 

Non-volatile memory comes in magnetic and optical forms, such as hard disk drives and CDs which use rotating disks that have particles on the surface which have been magnetised to encode data, or that have reflective portions that have been marked by lasers; or by using semiconductors to mechanically address memory. Semiconductor memory includes the majority of types of memory that are not optical or magnetic, such as Flash Memory, and Read Only Memory (ROM). ROM is a very basic form of non-volatile memory, where it is very difficult or impossible to modift data stored. Programmable ROM (PROM) is a developed ROM, which is manufactured blank, and can be programmed permanantly using specialised programming software. Erasable PROM (EPROM) is a further improvement to PROM, the only difference being that once programmed, the chip can be exposed to UV radiation which dissipates charge from gates, removing memory, allowing the chip to be programmed again. An evolution of EPROM is Electrical EPROM (EEPROM), which allows the data to be erased and reprogrammed in-circuit, a single-byte at a time. Flash Memory is a development of EEPROM, allowing memory to be erased and reprogrammed blocks at a time, and is therefore much faster than EEPROM.

