\section{Power Supply}
\subsection{Voltage Measurement}\label{volt-meas}
Electrical potential or voltage is measured in a wide range of sensing applications, as it is readily converted into digital signals for processing. This is undertaken through the use of Analogue to Digital Converters (ADCs). There are various types of ADC, each of which has associated benefits and drawbacks.

Flash ADCs use a sequence of comparators and logic gates to facilitate an extremely fast conversion. Hhowever, the number of output bits, the number of logic gates and comparators needed increases exponentially because one comparator is needed for each discrete value. Therefore the cost implications of flash ADCs of >8bits are prohibitive for many applications.

Successive approximation ADCs utilise a comparator coupled with a digital-to-analogue converter to make increasingly accurate digital values over a series of attempts. Because more attempts are needed to convert different analogue voltages, the conversion duration is not fixed. These ADCs can be produced with much higher resolution, at the detriment of sample rate.

Sub-ranging ADCs are a hybrid of flash and successive approximtion ADCs. The more significant bits are determined by a successive approximation technique, and the lower bits converted by a flash technique. This gives the ADCs some of the speed of an all-flash ADC without the cost implications, whilst retaining the resolution of successive approcimation ADCs.

Often, the ADCs available have strict voltage ranges, and signal conditioning is needed before the voltage applied is safe to measure. This can consist of simple voltage dividers or operational amplifers, which can be arranged in a plethora of configurations. The most universal application is the instrumentation amplifier \cite{instr-amp}, which measures the differential voltage between two input terminals, with an adjustable gain.


\subsection{Voltage Control}
Voltage control can be carried out using digital to analogue converters (DACs), which take in a digital signal and convert the value into an analogue voltage. 

The simplest of DAC techniques is pulse width modulation (PWM). The output voltage is switched on and off at a set frequency, and the pulse duration is adjusted according to the digital input. By passing the output signal through a low pass filter, the waveform is smoothed toward a perfect DC voltage, proportional to the digital input. The advantage of PWM is that it is cheap to implement - existing power switching electronics can be used. However, the output voltage fluctuates which makes PWM unsuitable for high stability applications.

Another simple DAC is the R-2R ladder, which utilises a repeated pattern of resistors to generate an analogue voltage from a series of bits. Resistors are chosen so that the ``rung'' resistors are precisely twice the value of the ``rail'' resistors. Alternatively, additional bits can be added by matching the Thevenin equivalent reistance of the lower bits. R-2R ladders are low cost due to the availability of precision reistors, however their acuracy can be low if poorly matched resistors are used. *** Probably needs a daigram ***

Similar to the R-2R ladder, is the binary weighted DAC which uses precision internal voltages for each individual bit, combined with a summing circuit to produce the output voltage. These DACs are very fast but precision can suffer if the more significant bits' voltages are inacurate.


The thermometer-coded DAC uses a precision current source for each discrete value in its voltage range. This has the advantage of superb precision and speed, however costs escalate exponentially with additional bits of precision.

An alternative DAC technique is oversampling
***http://www.analog.com/static/imported-files/tutorials/MT-017.pdf***

*** Something about pulse/pulse reverse - \cite{Chandrasekar} ***


\subsection{Current Measurement}
Electrical current is the rate of flow of electrons. In Direct Current (DC) applications such as electroforming, this flow is one-way and directly relates to the flow of ions through the electrolysis medium. Measuring current is therefore key to calculating the mass transfer during the electroforming process.

The most basic form of current measurement can be undertaken using a sub-ohm ``current sense'' resistor of known value. The resistor is placed in series with the circuit under test, and the voltage drop across it is measured using one of the techniques discussed in \ref{volt-meas}. The current through the resistor, and therefore the circuit under test (see Kirchoff's Current Law \cite{kirchoff}), can be calculated using Ohm's Law \cite{ohm} $V = I \times R$. The benefit of using a small resistance value is that the power dissipated is low. For stability of measurement, it is important that the temperature coefficient of resistance (TCR) of the resistor is low, so that it is unaffected by resistive heating. 

An alternative to the current sense resistor is to use an inductor's DC resistance. An idea inductor has no DC resistance, however in reality they exhibit resistances of sub-milliohm levels. This means that the power loss is much less than in the current sense reistor method, however due to the high TCR of copper, the current measurement varies greatly with temperature.

Another method of current sensing is to use the power control transistors already used in the circuit. This has the benefit of requiring no additional power-disipating devices in the supply line. The internal ``on resistance'' of the transistor, which can be found in the part's datasheet, is used to provide the voltage drop to measure. Transistors often have extremely large TCRs, and therefore the accuracy of measurement cannot be guaranteed. 

Some transistors are designed with built in ``ratiometric current sense transistors'', which use a portion of the transistor to create and isolated sense transistor. When the transistor is on, the current through the sense transistor will be a ratio of the overall current. Depending upon the model used, the accuracy can range from 5\% to 20\%, making them unsuitable for precise current control applications.

Alternatively, current can be measured indirectly using the Hall effect \cite{hall}. When a current carying conductor is placed in an electric field, there is a potential induced perpendicular to the current flow. This potential is porportional to the current, and can be measured using one of the techniques discussed in \ref{volt-meas}. However, this method of sensing is not applicable to low current applications as the induced Hall voltage is weak. Hall sensors are also suceptible to interference from external megnetic fields and non-linear temperature drift, making them unsuitable for mny applications. However, Hall sensors have the advantage of low power dissipation, and are capable of measuring high currents.

\subsection{Current Control}