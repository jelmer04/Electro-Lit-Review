\section{Power Supply}
\subsection{Voltage Measurement}\label{volt-meas}
Electrical potential or voltage is measured in a wide range of sensing applications, as it is readily converted into digital signals for processing. This is undertaken through the use of Analogue to Digital Converters (ADCs). There are various types of ADC, each of which has associated benefits and drawbacks.

Flash ADCs use a sequence of comparators and logic gates to facilitate an extremely fast conversion. Hhowever, the number of output bits, the number of logic gates and comparators needed increases exponentially because one comparator is needed for each discrete value. Therefore the cost implications of flash ADCs of >8bits are prohibitive for many applications.

Successive approximation ADCs utilise a comparator coupled with a digital-to-analogue converter to make increasingly accurate digital values over a series of attempts. Because more attempts are needed to convert different analogue voltages, the conversion duration is not fixed. These ADCs can be produced with much higher resolution, at the detriment of sample rate.

Sub-ranging ADCs are a hybrid of flash and successive approximtion ADCs. The more significant bits are determined by a successive approximation technique, and the lower bits converted by a flash technique. This gives the ADCs some of the speed of an all-flash ADC without the cost implications, whilst retaining the resolution of successive approcimation ADCs.


\subsection{Voltage Control}


\subsection{Current Measurement}
Electrical current is the rate of flow of electrons. In Direct Current (DC) applications such as electroforming, this flow is one-way and directly relates to the flow of ions through the electrolysis medium. Measuring current is therefore key to calculating the mass transfer during the electroforming process.

The most basic form of current measurement can be undertaken using a sub-ohm resistor of known value. The resistor is placed in series with the circuit under test, and the voltage drop across it is measured using one of the techniques discussed in \ref{volt-meas}. The current through the resistor, and therefore the circuit under test (see Kirchoff's Current Law \cite{}), can be calculated using Ohm's Law \cite{} $V = I \times R$.


\subsection{Current Control}