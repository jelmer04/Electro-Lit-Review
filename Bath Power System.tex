\section{Voltage Measurement}
Electrical potential or voltage is measured in a wide range of sensing applications, as it is readily converted into digital signals for processing. This is undertaken through the use of Analogue to Digital Converters (ADCs). There are various types of ADC, each of which has associated benefits and drawbacks.

Flash ADCs use a sequence of comparators and logic gates to facilitate an extremely fast conversion. Hhowever, the number of output bits, the number of logic gates and comparators needed increases exponentially because one comparator is needed for each discrete value. Therefore the cost implications of flash ADCs of >8bits are prohibitive for many applications.

Successive approximation ADCs use a comparator coupled with a digital-to-analogue converter to get a gradually more accurate digital value. Successive approximation ADCs take a different amount of time to arrive at a digital value depending upon how many successive steps are taken to reach the correct analogue value. These ADCs can be produced with much higher resolution, but at lower sample rates.

Sub-ranging ADCs use successive approximation techniques to find the most significant bit values, and a flash ADC to find the least significant bits. This allows sub-ranging ADCs to perform faster than successive approximation ADCs, but slower than flash ADCs. The ability to obtain the most significant bits by successive approximation allows these ADCs to have high resolution, without the vastly increased cost of purely flash ADCs.




\section{Voltage Control}


\section{Current Measurement}


\section{Current Control}