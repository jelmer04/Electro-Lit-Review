
\subsection{Collecting and Storing Data}
\subsubsection{Memory}
Different techniques can be used for storing digital data, which can be split into two categories. Volatile memory requires constant power to maintain saved data, but if the power is interrupted, the data is lost. Volatile memory is very fast to save to and recover from, but due to the need for constant power, is not used when data is to be stored for long periods of time. Non-volatile memory is not lost when power to the memory device is removed, and is therefor a much more reliable method of storing important data. 

Non-volatile memory comes in magnetic and optical forms, such as hard disk drives and CDs which use rotating disks that have particles on the surface which have been magnetised to encode data, or that have reflective portions that have been marked by lasers; or by using semiconductors to mechanically address memory. Semiconductor memory includes the majority of types of memory that are not optical or magnetic, such as Flash Memory, and Read Only Memory (ROM). ROM is a very basic form of non-volatile memory, where it is very difficult or impossible to modify data stored. Programmable ROM (PROM) is a developed ROM, which is manufactured blank, and can be programmed permanently using specialised programming software. Erasable PROM (EPROM) is a further improvement to PROM, the only difference being that once programmed, the chip can be exposed to UV radiation which dissipates charge from gates, removing memory, allowing the chip to be programmed again. An evolution of EPROM is Electrical EPROM (EEPROM), which allows the data to be erased and reprogrammed in-circuit, a single-byte at a time. Flash Memory is a development of EEPROM, allowing memory to be erased and reprogrammed blocks at a time, and is therefore much faster than EEPROM.

For data-logging, a non-volatile memory should be used which can be accessed and edited quickly, making EEPROM or Flash Memory the most appropriate method of storing data. 

\subsection{Hardware}
Data acquisition (DAQ) can be executed in several ways, but the overall process is generally that a sensor converts continuous (analogue) data from a measurand into an electrical (digital) signal, which is sent to a display or actuator. Data logging is a way of displaying and recording the information, and data loggers or data acquisition systems are the hardware between the sensor and the computer.

Parameters which are generally measured include temperature, voltage, current, strain, power, pressure, acceleration and resistance, and depending upon the need, data logging hardware is available to accept just one channel for one measurement, or multiple channels so several measurements can be taken at once. 

Single channel data loggers are usually stand-alone, and data is not read by a PC in real time, but rather when logging is completed, the memory device in the data logger is inserted into and read by the PC. Data loggers which use only internal memory like this, as opposed to those which transmit data in real time to a computer, usually have a low sample rate due to the limited memory.

Multiple channel data loggers transfer data to a PC easily by the use of buses such as USB (Universal Serial Bus) or ethernet, or wirelessly. USBs have the advantage that they are readily accepted and auto detected by PCs, however the cable length is usually limited to 3 to 5m. Ethernet cables require more installation effort, but can be used over a much larger range from sensor to PC, up to 100m [www.ni.com], and can connect data straight to a network to be accessed by several PCs at once. In some cases, cables may be inconvenient, either because they would have to run a long distance, or due to time spent installing the cables, and determining the most efficient infrastructure, so wireless data transfer has an advantage; however a disadvantage is that data transfer is much slower and less reliable with wireless transfer.

For the purpose of reliable, real time data acquisition on a PC, from an electro-forming system, ethernet data transfer is the most advantageous method; having the data accessible from several PCs secures data storage, and the system is small and compact, so wireless technology would not be necessary. 
 
\subsection{Analysing Data}  ((software))

\subsection{User Interface}

