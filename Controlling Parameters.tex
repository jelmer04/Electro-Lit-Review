\section{Controlling Parameters}
\subsection{Intro}
In order to control parameters in the electro-forming system, a control system must be employed, using feedback to ensure successful control. The system should allow the user to manually enter required set point values for each parameter.

Sensors should be employed to measure parameters of interest in the electro-forming system, which will be acquired and processed in data acquisition systems. The data acquisition module will pass the data onto a computer or micro-controller (hereafter know as PC), which can store the data in forms which are convenient such as graphs, and can determine the difference between the measured value in comparison to the set point.The PC will then pass on control signals to actuators, which will operate accordingly to reduce the error between the measured value and set point. 
